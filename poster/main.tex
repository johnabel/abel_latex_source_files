\documentclass{beamer}
\usepackage[orientation=landscape,width=48in,height=36in,scale=1.4,debug]{beamerposter}
\mode<presentation>{\usetheme{MGH}}
\usepackage[scaled]{helvet}
\usepackage{chemformula}
\usepackage{siunitx} %pretty measurement unit rendering
\usepackage{ragged2e}
\usepackage[font=scriptsize,justification=justified]{caption}
\usepackage{array,booktabs,tabularx}

\newcommand<>{\fullpagegraphic}[2][1.0]{
  \only#3{
  \dimen1=#1\textwidth\relax
  \dimen2=#1\textheight\relax
  \dimen3=0.8\dimen2\relax
  \begin{center}
    \includegraphics[width=\dimen1,height=\dimen3,keepaspectratio]{#2}
  \end{center}
}}

\usepackage{tcolorbox}
\tcbuselibrary{breakable}
\tcbuselibrary{skins}
\newenvironment{examplebox}[1]{\begin{center}\begin{tcolorbox}[colback=green!5!white,colframe=green!75!black,width=\textwidth,title={#1},breakable,fonttitle=\color{black}\small]}{\end{tcolorbox}\end{center}}

\renewcommand{\baselinestretch}{0.8}

\title{\textbf{\LARGE TITLE}}
\author{Author List}
\institute[MIT]{Affiliations List}
\date{ }

% edit this depending on how tall your header is. We should make this scaling automatic :-/
\newlength{\columnheight}
\setlength{\columnheight}{69cm}

\begin{document}
\begin{frame}
\begin{columns}
	\column{0.25\textwidth}
		\begin{beamercolorbox}[center]{postercolumn}
			\begin{minipage}{.98\textwidth}  % tweaks the width, makes a new \textwidth
				\parbox[t][\columnheight]{\textwidth}{ % must be some better way to set the height, width and textwidth simultaneously
					\begin{myblock}{Abstract}
						{\footnotesize 
		Lorem ipsum
						}
						\begin{examplebox}{Research questions and aims}
											{\small\color{black}
						\begin{enumerate}
							\item What role do AVP neurons play in SCN communication?\\[0.5em]
							\item How do AVP and VIP pathways interact?\\[0.5em]
							\item Develop experimental and computational techniques for analyzing cell-type-specific circadian oscillation.
						\end{enumerate}
					}
					\end{examplebox}
					\end{myblock}

					\begin{myblock}{Color-switching PER2::iLUC reporter}
						
					\end{myblock}
					\vfill
					}
			\end{minipage}
		\end{beamercolorbox}

	\column{0.5\textwidth}
		\begin{beamercolorbox}[center]{postercolumn}
			\begin{minipage}{.98\textwidth}  % tweaks the width, makes a new \textwidth
				\parbox[t][\columnheight]{\textwidth}{ % must be some better way to set the height, width and textwidth simultaneously
					\begin{myblock}{Computational analysis of cellular circadian oscillation}
						
					\end{myblock}\vfill
					}
			\end{minipage}
		\end{beamercolorbox}

	\column{0.25\textwidth}
			\begin{beamercolorbox}[center]{postercolumn}
			\begin{minipage}{.98\textwidth}  % tweaks the width, makes a new \textwidth
				\parbox[t][\columnheight]{\textwidth}{ 
				% must be some better way to set the height, width and textwidth simultaneously
				
					\begin{myblock}{Proposed mechanism of AVP-VIP signaling}

					\end{myblock}

					\begin{myblock}{Conclusions}

					\end{myblock}

					\begin{myblock}{References and acknowledgement}
						{\footnotesize
							\begin{enumerate}
								\item Example
						\end{enumerate}

						{\small
					Research was supported by NIH training award T32-HLO9701 (JHA), the Howard Hughes Medical Institute (JST), and NIH/NINDS grant R01 NS106657 (JST).
					}
						}
					\end{myblock}\vfill
					}

					
			\end{minipage}
		\end{beamercolorbox}

\end{columns}
\end{frame}
\end{document}
