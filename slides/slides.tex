\documentclass[11.5pt]{beamer}
\usetheme{nasa}
\metroset{block=fill}


% for printing handouts
%\documentclass[handout,12pt]{beamer}
%\usepackage{pgfpages}
%\usepackage{tikz}
%\pgfpagesuselayout{4 on 1}[a4paper, landscape, border shrink=5mm]
%\setbeamercolor{background canvas}{bg=white}
%\setbeamercolor{normal text}{fg=black,bg=white}
%\setbeamercolor{titlelike}{fg=black,bg=white}
%\setbeamertemplate{frametitle}{\color{black}\bfseries\insertframetitle\par}
%% end handouts



% % %
% J H Abel custom commands here. not in nasa theme so they can be used to
% print slides
% % %

% a different kind of section
\newcommand{\nasasection}[1]{
    \metroset{background=dark}\section{#1}\metroset{background=light}
}

% font theme for math: serif
\usefonttheme[onlymath]{serif}

% fullpagegrphic command
\newcommand<>{\fullpagegraphic}[2][1.0]{
  \only#3{
  \dimen1=#1\textwidth\relax
  \dimen2=#1\textheight\relax
  \dimen3=0.8\dimen2\relax
  \begin{center}
    \includegraphics[width=\dimen1,height=\dimen3,keepaspectratio]{#2}
  \end{center}
}}

% footlineextra using tikz
\newcommand{\footlineextra}[1]{
    \begin{tikzpicture}[remember picture,overlay]
        \node[yshift=2ex,anchor=south west] at (current page.south west) {\usebeamerfont{author in head/foot}\hspace{1ex}\tiny#1};
    \end{tikzpicture}
}
% locations of figures
\graphicspath{{./figures/}}

% % %
% End of preamble. set up the doc.
% % %




% title etc.
\title{Neurobiology of Circadian Rhythms}
\author[Abel - Harvard Medical School]{John Abel Ph.D.}
\date{3 June 2019}
\institute{
Division of Sleep Medicine, Harvard Medical School\\
Department of Anesthesiology, Massachusetts General Hospital\\
Picower Institute for Learning and Memory, MIT}


\begin{document}

\metroset{background=dark}
\maketitle
\metroset{background=light}


\begin{frame}{Learning objectives}
\begin{enumerate}
\item Learn the three essential charactertistics of circadian rhythms.\pause
\item Understand how mammalian circadian rhythms are generated within a single cell and how these rhythms are coordinated.\pause
\item Understand the signals exhanged within the hypothalamus, brain, and peripheral tissues.
\end{enumerate}
\end{frame}



\begin{frame}{Outline}
  \setbeamertemplate{section in toc}[sections numbered]
  \tableofcontents[hideallsubsections]
\end{frame}


\nasasection{An Introduction to Circadian Rhythms}

\begin{frame}{What are circadian rhythms?}
    \begin{columns}
    \column{0.5\textwidth}
    \fullpagegraphic{cassia_corymbosa.jpg}
    \footlineextra{\textit{Cassia corymbosa}, sketch by Darwin}

    \column{0.5\textwidth}
    \begin{itemize}
        \item Endogenous, entrainable, near-24h oscillations in gene expression, metabolism, or behavior
    \end{itemize}
    \end{columns}
\end{frame}
\begin{frame}[noframenumbering]{What are circadian rhythms?}
    \begin{columns}
    \column{0.5\textwidth}
    \fullpagegraphic{cassia_corymbosa.jpg}
    \footlineextra{\textit{Cassia corymbosa}, sketch by Darwin}

    \column{0.5\textwidth}
    \begin{itemize}
        \item {\bf Endogenous}, entrainable, near-24h oscillations in gene expression, metabolism, or behavior
    \end{itemize}
    \end{columns}
\end{frame}
\begin{frame}[noframenumbering]{What are circadian rhythms?}
    \begin{columns}
    \column{0.5\textwidth}
    \fullpagegraphic{cassia_corymbosa.jpg}
    \footlineextra{\textit{Cassia corymbosa}, sketch by Darwin}

    \column{0.5\textwidth}
    \begin{itemize}
        \item Endogenous, {\bf entrainable}, near-24h oscillations in gene expression, metabolism, or behavior
    \end{itemize}
    \end{columns}
\end{frame}
\begin{frame}[noframenumbering]{What are circadian rhythms?}
    \begin{columns}
    \column{0.5\textwidth}
    \fullpagegraphic{cassia_corymbosa.jpg}
    \footlineextra{\textit{Cassia corymbosa}, sketch by Darwin}

    \column{0.5\textwidth}
    \begin{itemize}
        \item Endogenous, entrainable, {\bf near-24h} oscillations in gene expression, metabolism, or behavior
    \end{itemize}
    \end{columns}
\end{frame}
\begin{frame}[noframenumbering]{What are circadian rhythms?}
    \begin{columns}
    \column{0.5\textwidth}
    \fullpagegraphic{cassia_corymbosa.jpg}
    \footlineextra{\textit{Cassia corymbosa}, sketch by Darwin}

    \column{0.5\textwidth}
    \begin{itemize}
        \item Endogenous, entrainable, near-24h oscillations in gene expression, metabolism, or behavior
        \item (\bf{temperature-compensated})
    \end{itemize}
    \end{columns}
\end{frame}

\begin{frame}{Circadian rhythms are ubiquitous}
    \fullpagegraphic{cyano.JPG}
    \footlineextra{Wikimedia Commons}
\end{frame}
\begin{frame}{Circadian rhythms are ubiquitous}
    \fullpagegraphic{konopka1972.png}
    \footlineextra{Konopka and Benzer PNAS 1972}
\end{frame}
\begin{frame}{Circadian rhythms are ubiquitous}
    \footlineextra{(L) van der Horst \textit{at al.} Nature 1999, (R) Takahashi Nat Rev Genet 2017}
    \begin{columns}
        \column{0.4\textwidth}
    \fullpagegraphic{vanderhorst.png}
    \pause
    \column{0.6\textwidth}
        \fullpagegraphic{circadian_landscape.png}
    \end{columns}
\end{frame}


% where do these rhythms come from? how are they so similar?
\nasasection{The Master Clock Hypothesis}


\begin{frame}{Physiology of the suprachiasmatic nucleus (SCN)}
\footlineextra{fig: Alila Medical Media}

\begin{columns}
\column{0.5\textwidth}
\fullpagegraphic{alila.png}
\column{0.5\textwidth}
\begin{itemize}
\item located in the hypothalamus
\item bilateral of the 3rd ventricle
\item ventral SCN receives light inputs from retinohypothalamic tract
(RHT)
\end{itemize}
\end{columns}
\end{frame}


\begin{frame}{Evidence for SCN role in circadian behavior}
\footlineextra{Stephan and Zucker PNAS 1972}

SCN lesioning eliminates behavioral rhythms drinking in
rats.

\fullpagegraphic{stephan_zucker.png}
\end{frame}


\begin{frame}{SCN transplants determine circadian period}
    \footlineextra{Ralph \textit{et al.} Science 1990}

    \vspace{-1cm}
    \fullpagegraphic{transplant.png}
\end{frame}


\begin{frame}{The master clock}
``Master clock'' hypothesis: SCN keeps time and drives rhythms in other neural and peripheral tissue.

\fullpagegraphic{melatonin.png}
\footlineextra{Czeisler \textit{et al.} NEJM 1995}

\end{frame}


\begin{frame}{The oscillator in the SCN is cell-autonomous}
\footlineextra{Welsh \textit{at al.} Neuron 1995}

\vspace{-1cm}
\fullpagegraphic{welsh1995.png}
\end{frame}

\begin{frame}

{\huge Where do these oscillations originate?}\\[1em]
To answer this question, we need to look at the contemporary non-mammalian research.
\end{frame}

\nasasection{The Genetic Oscillator and the Master Pacemaker Hypothesis}


\begin{frame}{Circadian rhythms in drosophila}
    \fullpagegraphic{konopka1972.png}
    \footlineextra{Konopka and Benzer PNAS 1972}
\end{frame}


\begin{frame}{Genetic origins of mammalian circadian rhythm}
    \begin{columns}
    \column{0.4\textwidth}
    1990s: Identification and cloning of the \textit{CLOCK} gene in mammals
    \column{0.6\textwidth}
    \fullpagegraphic{vitaterna.png}
    \end{columns}
    \footlineextra{Vitaterna \textit{et al.} 1994}
\end{frame}


\begin{frame}{Genetic origins of mammalian circadian rhythm}
    How can a genetic circuit drive oscillatory gene expression? What about other non-transcriptional mechanisms, like those in cyanobacteria?
\end{frame}


\begin{frame}{Biological feedback loops drive oscillation}
    \fullpagegraphic{loop0.png}
    \footlineextra{Example: a highly simplified genetic feedback loop.}
\end{frame}
\begin{frame}[noframenumbering]{Biological feedback loops drive oscillation}
    \fullpagegraphic{loop1.png}
    \footlineextra{Example: a highly simplified genetic feedback loop.}
\end{frame}
\begin{frame}[noframenumbering]{Biological feedback loops drive oscillation}
    \fullpagegraphic{loop2.png}
    \footlineextra{Example: a highly simplified genetic feedback loop.}
\end{frame}
\begin{frame}[noframenumbering]{Biological feedback loops drive oscillation}
    \fullpagegraphic{loop3.png}
    \footlineextra{Example: a highly simplified genetic feedback loop.}
\end{frame}
\begin{frame}[noframenumbering]{Biological feedback loops drive oscillation}
    \fullpagegraphic{loop4.png}
    \footlineextra{Example: a highly simplified genetic feedback loop.}
\end{frame}
\begin{frame}[noframenumbering]{Biological feedback loops drive oscillation}
    \fullpagegraphic{loop5.png}
    \footlineextra{Example: a highly simplified genetic feedback loop.}
\end{frame}


\begin{frame}{Mathematical structure of a circadian oscillator}
    \begin{columns}
    \column{0.5\textwidth}
    Biochemical states $x(t)$ are governed by dynamics:
    \begin{equation*}
        \frac{dx}{dt} = f(x, p, u).
    \end{equation*}
    \pause

    \vfill
    The system has an $n$-dimensional attractive limit cycle, meaning
    \begin{equation*}
        \lim_{t\to\infty}[x(t) - x(t-T)]=0,
    \end{equation*}

    period: $T$\pause
    \column{0.5\textwidth}
    \fullpagegraphic{state_space.pdf}
    \footlineextra{Fig: PC St.\ John}
    \end{columns}
\end{frame}


\begin{frame}{Limit cycle phase responses}
    A perturbation shifts the oscillator away from the limit cycle.

    \fullpagegraphic{phase_response0.pdf}
\end{frame}
\begin{frame}[noframenumbering]{Limit cycle phase responses}
    A perturbation shifts the oscillator away from the limit cycle.

    \fullpagegraphic{phase_response1.pdf}
\end{frame}
\begin{frame}[noframenumbering]{Limit cycle phase responses}
    System dynamics govern its eventual return.

    \fullpagegraphic{phase_response2.pdf}
\end{frame}
\begin{frame}[noframenumbering]{Limit cycle phase responses}
    The oscillator returns to the cycle with a new phase (advance).

    \fullpagegraphic{phase_response3.pdf}
\end{frame}

\begin{frame}{Phase response curves (PRCs)}

    Describes how the system responds to a perturbation of a certain strength and duration applied at each phase.

    \begin{center}
    \includegraphics[height=2.5in]{humanprc.png}
\end{center}

    \footlineextra{Kronauer, Forger, Jewett, J Biol Rhythms 1999}
\end{frame}


\begin{frame}{Evidence for a physical limit cycle underlying\\
circadian oscillation}
    \begin{enumerate}[<+->]
    \item Self-sustained oscillation
    \item Amplitude dynamics
    \item Phase shifts following perturbation
    \item Gene networks mimic mathematical structure
    \end{enumerate}
\end{frame}


\begin{frame}{Structure of the\\ mammalian genetic\\ oscillator}

\begin{columns}
\column{0.3\textwidth}
\column{0.7\textwidth}
\vspace{-1.5cm}
\includegraphics[height=\textheight]{full_network.png}
\end{columns}

\footlineextra{Takahashi Nat Rev Genet 2017}
\end{frame}


\begin{frame}{Structure of the\\ mammalian genetic\\ oscillator}

\begin{columns}
\column{0.3\textwidth}
\column{0.7\textwidth}
\vspace{-1.5cm}
\includegraphics[height=\textheight]{negative.png}
\end{columns}

\footlineextra{Takahashi Nat Rev Genet 2017}
\end{frame}


\begin{frame}{Structure of the\\ mammalian genetic\\ oscillator}

\begin{columns}
\column{0.3\textwidth}
\column{0.7\textwidth}
\vspace{-1.5cm}
\includegraphics[height=\textheight]{positive.png}
\end{columns}

\footlineextra{Takahashi Nat Rev Genet 2017}
\end{frame}


\begin{frame}{Structure of the\\ mammalian genetic\\ oscillator}

\begin{columns}
\column{0.3\textwidth}
\column{0.7\textwidth}
\vspace{-1.5cm}
\includegraphics[height=\textheight]{outputs.png}
\end{columns}

\footlineextra{Takahashi Nat Rev Genet 2017}
\end{frame}


\begin{frame}{Local oscillators regulate metabolic processes}

\includegraphics[height=0.7\textheight]{metabolismmeter.png}

Deep integration of circadian regulation with cellular metabolism.

\footlineextra{Green, Takahashi, and Bass, Cell 2008}
\end{frame}


\begin{frame}{Local oscillators regulate metabolic processes}

\fullpagegraphic{circadian_landscape.png}

10-20\% of all transcripts under circadian regulation.

\footlineextra{Takahashi Nat Rev Genet 2017}
\end{frame}


\begin{frame}{Recording from biological feedback loops}
    \fullpagegraphic{biolumfeedback_loop.png}
    \footlineextra{Yoo \textit{et al.} PNAS 2004}
\end{frame}


\begin{frame}{Luciferase reporter shows clock is deeply integrated in metabolism}

\fullpagegraphic{zhang_rnai.png}


\footlineextra{Zhang \textit{et al.} Cell 2009}
\end{frame}


\begin{frame}{PER2::Luc reporter and the master clock hypothesis}
    \fullpagegraphic{per2luc.png}
    \footlineextra{Yoo \textit{et al.} PNAS 2004}
\end{frame}


\begin{frame}{Revisiting understanding of the SCN role}
    \fullpagegraphic{per2luc2.png}
    \footlineextra{Yoo \textit{et al.} PNAS 2004}
\end{frame}


\begin{frame}{Master clock vs. master pacemaker}
    
    Yoo \textit{et al.} provided evidence \textbf{against} the master clock hypothesis--there are many clocks.\\[1em]
    \pause

    Now, the SCN appears to be a ``master pacemaker,'' setting the time and coordinating peripheral oscillators in local tissue.
\end{frame}


\begin{frame}{Full SCN tissue sustains synchronized rhythms}
    \fullpagegraphic{scn.png}
    \footlineextra{Herzog \textit{et al.} Meth Enzymol 2015}

\end{frame}


\begin{frame}{Full SCN tissue sustains rhythms}
    \fullpagegraphic{sync.png}
    \footlineextra{Herzog \textit{et al.} Meth Enzymol 2015}

\end{frame}



\begin{frame}{Neural oscillators must communicate to be effective}
    \fullpagegraphic{herzog2004c.png}
    \footlineextra{Herzog \textit{et al.} J Biol Rhythms 2004}

    \pause
    If intercellular communication is interrupted, oscillators lose synchrony and amplitude.

\end{frame}


\begin{frame}
    Peripheral tissues must receive input to remain aligned to the SCN.\\\pause
    Neurons must exchange signals to remain synchronized.\\\pause
    The SCN must receive input to remain aligned to the environment.\pause\\[1em]

    {\huge How are these processes coordinated across cells and tissues?}
\end{frame}



\nasasection{Circadian Signals Inside the SCN}




\begin{frame}{Basics of neurotransmission}
    \fullpagegraphic{nts.jpg}
    \footlineextra{Fig: NIH}
\end{frame}


\begin{frame}{Neural signals: terminology and review}
Inhibitory: signal makes a neuron less likely to fire\\
Excitatory: signal makes a neuron more likely to fire\\[1em]


\textbf{Ionotropic} neurotransmitters cause immediate inhibitory or
excitatory input.\\
\textbf{Metabotropic} neurotransmitters affect the cell
metabolism, and can be inputs to the circadian clock to affect
oscillation.
\end{frame}

\begin{frame}{Pathways of neurotransmission in the SCN}

    \begin{columns}
    \column{0.6\textwidth}
    \fullpagegraphic{allsignals.png}

    \column{0.4\textwidth}
    \begin{itemize}
        \item Firing drives release of metabotropic neurotransmitters
        \item Neurotransmitters modulate circadian transcription via CREB
    \end{itemize}

    \end{columns}

    \footlineextra{Welsh, Takahashi, Kay Annu Rev Physiol 2010}
\end{frame}


\begin{frame}{Spatial organization of the SCN}
\fullpagegraphic{SCN_localization.png}

\footlineextra{Moore and Silver Chronobiol Int 1998}

\end{frame}


\begin{frame}{Question: which neurons are implicated in synchrony?}
    \footlineextra{Abel \textit{et al.} PNAS 2016}

    \vspace{-1cm}
    Experiment: block electrical firing, wait for desynchrony, allow to resynchronize. Infer connectivity with mutual information.
    \pause

    \begin{columns}
        \column{0.3\textwidth}

        \fullpagegraphic{luminescent.png}

        PER2::LUC SCN
        
        \small{collaborator: Daniel Granados-Fuentes, Herzog lab}\pause
        \column{0.7\textwidth}
        \fullpagegraphic{resync.png}

    \end{columns}


\end{frame}

\begin{frame}{Structure of the SCN network}
    \footlineextra{Abel \textit{et al.} PNAS 2016}

    \begin{center}
        \includegraphics[height=1.5in]{fig5_new.pdf}
    \end{center}

    Small-world, bilateral symmetric, with hubs located in SCN central regions.\\\pause
    This does not perfectly correlate to a single known distribution of neurotransmitter release. Which are involved?

\end{frame}


\begin{frame}{The role of VIP neurons in synchrony and entrainment}
    \footlineextra{Aton \textit{et al}. Nat Neurosci 2005}
    \begin{columns}
        \column{0.5\textwidth}
        One pathway likely driving this synchrony is VIP.\\[0.5em]

        \pause

    It is unknown how VIP is released endogenously, and how this is dependent upon electrical activity.
    \column{0.5\textwidth}

    \pause
    \fullpagegraphic{VIP.png}
    \end{columns}
\end{frame}


\begin{frame}{The role of VIP neurons in synchrony and entrainment}
    \footlineextra{Aton \textit{et al}. Nat Neurosci 2005}
    \begin{columns}
        \column{0.5\textwidth}
        One pathway likely driving this synchrony is VIP.\\[0.5em]


    It is unknown how VIP is released endogenously, and how this is dependent upon electrical activity.
    \column{0.5\textwidth}

    \fullpagegraphic{vpac2r_agonist.png}
    \end{columns}
\end{frame}


\begin{frame}{Study design: investigating role of VIP, electrical neurotransmission, and synchrony}

    \begin{enumerate}[<+->]
        \item Identify firing patterns in the SCN.
        \item Apply those patterns to VIP neurons via optogenetic stimulation.
        \item Characterize VIP release and circadian response.
    \end{enumerate}

    \footlineextra{Mazuski, Abel, Chen, Hermanstyne, Doyle, and Herzog, Neuron 2018}
\end{frame}

\begin{frame}{Recording from VIP and non-VIP SCN neurons}

        \includegraphics[height=1.2in]{expt_setup.png}\\\pause
        \includegraphics[height=1.2in]{expt_record.png}

    \footlineextra{Mazuski, Abel, ..., Doyle, and Herzog, Neuron 2018}
\end{frame}


\begin{frame}{Identifying firing patterns in the SCN}

        \fullpagegraphic{mazuski2.png}

    \footlineextra{Mazuski, Abel, ..., Doyle, and Herzog, Neuron 2018}
\end{frame}

\begin{frame}{Optogenetic stimulation triggers VIP release in a frequency-dependent fashion}

    \begin{columns}
        \column{0.5\textwidth}
    Apply optogenetic stimulation with constant rate at variable instantaneous frequency.\pause

    \fullpagegraphic{Figure3a.png}\pause

    \column{0.5\textwidth}
    VIP antagonist eliminates phase shift.

    \fullpagegraphic{Figure3c.png}
    \end{columns}
\end{frame}

\begin{frame}{Optogenetic stimulation of VIP neurons entrains circadian behavior in vivo via phase delays}
    \fullpagegraphic{entrainment.png}
    \footlineextra{Mazuski, Abel, ..., Doyle, and Herzog, Neuron 2018}

\end{frame}


\begin{frame}{A new potential tool: Cre-inducible color-switching}

\fullpagegraphic{iamging.png}

\footlineextra{Shan, Abel,..., Doyle III, Takahashi in revision}
\end{frame}


\begin{frame}{The roles of VIP and AVP in the SCN}

\fullpagegraphic{avpbmal.png}

\footlineextra{Shan, Abel,..., Doyle III, Takahashi in revision}
\end{frame}


\begin{frame}{The roles of VIP and AVP in the SCN}

\fullpagegraphic{vipbmal.png}

\footlineextra{Shan, Abel,..., Doyle III, Takahashi in revision}
\end{frame}


\begin{frame}{Neurotransmission in the SCN}
    \begin{enumerate}
        \item Neurotransmission in the SCN relies on the confluence of numerous pathways and processes. 
            \begin{itemize}
                \item The central SCN is of importance within the network. \pause
                \item VIP primarily evokes phase delays.\pause
                \item New evidence suggests clocks in AVP neurons are essential for synchrony \pause
                \item Neuropeptide release depends on electrical activity.\pause
            \end{itemize}
        \item Electrical activity: a fundamental component of the oscillator
    \end{enumerate}
\end{frame}




\begin{frame}

{\huge How does the SCN integrate environmnetal signals and coordinate peripheral tissues?}
\end{frame}

\nasasection{Circadian Signals Outside the SCN}


\begin{frame}{Entrainment to light}

\begin{columns}
\column{0.5\textwidth}
Entrainment: the synchronization of biological processes to an external rhythm\pause
\column{0.5\textwidth}
\fullpagegraphic{robynhall.png}
\end{columns}

\footlineextra{fig: Robyn Hall, Twitter}
\end{frame}

\begin{frame}{Entrainment to light: physiology}
\begin{columns}
\column{0.4\textwidth}
{\small Melanopsin in intrinsically photosensitive retinal ganglion cells
(ipRGCs) responds to light, signal transmitted down RHT.}
\column{0.6\textwidth}
\fullpagegraphic{iprgcs.png}
\end{columns}


\end{frame}


\begin{frame}{Entrainment to light: process}
\begin{columns}
\column{0.5\textwidth}
\begin{enumerate}
\item Light is received by ipRGCs\pause
\item ipRCGs fire, signal transmitted down RHT\pause
\item Glutamate release triggers SCN neuron firing\pause
\item Neurotransmitters and neuropeptides of the SCN cause phase shifts (via CREB)\pause
\end{enumerate}
\column{0.5\textwidth}
\fullpagegraphic{SCNbrain.png}
\end{columns}
\end{frame}


\begin{frame}{Neural signals to peripheral tissue}
\begin{columns}
\column{0.3\textwidth}
{\small SCN connects to other brain regions, which use hormones,
temperature, neural signals to keep rest of body synchronized with
the SCN.}
\column{0.7\textwidth}
\fullpagegraphic{levi2010}
\end{columns}
\end{frame}


\begin{frame}{Feeding as an entraining factor}
Oscillators in the tissue such as the liver may be entrained and
synchronized by timing of feeding.

\begin{center}
\includegraphics[height=3cm]{liverclocks.png}
\end{center}

\footlineextra{Vujovic, Davidson, Menaker Am J Physiol Regul Integr Comp Physiol 2008}

\end{frame}


\begin{frame}{Feeding as an entraining factor: local master regulators}

\fullpagegraphic{activity-red-green.png}

\footlineextra{Shan, Abel,..., Doyle III, Takahashi in prep}
\end{frame}


\begin{frame}{Everything can be seen as a closed loop}
\fullpagegraphic{summary.png}
\footlineextra{Takahashi Nat Rev Genet 2017}
\end{frame}


\begin{frame}
    \frametitle{Neurobiological hierarchy of circadian rhythm}
\fullpagegraphic{summary_hastings.png}

\footlineextra{Hastings \textit{et al.} Nat Rev Neurosci 2018}
\end{frame}


\begin{frame}[noframenumbering]
    \frametitle{Acknowledgments: Formal Affiliations}
    {\footnotesize
    \begin{columns}
        \column{0.5\textwidth}
        {\it Collaborators}

    Prof. Erik Herzog (WUSTL)\\
    Prof. Joseph Takahashi (UTSW)\\
    Dr. Cristina Mazuski\\
    Dr. Daniel Granados-Fuentes\\
    Dr. Yongli Shan

    \column{0.5\textwidth}
    {\it Advisors}

    Prof. Frank Doyle (Harvard)\\
    Prof. Linda Petzold (UCSB)\\
    Prof. Elizabeth Klerman (HMS/BWH)
    \end{columns}

    \vspace{0.3cm}

    \begin{columns}
        \column{0.45\textwidth}
    Doyle lab, Harvard University\\
    \includegraphics[height = 3.cm]{doylelab.jpg}

    \column{0.5\textwidth}
    AMU, Brigham and Women's Hospital\\
    \includegraphics[height = 3.cm]{amu.jpg}
    \end{columns}

    \vfill
    {\it Funding and affiliations:}\hspace{0.5cm}
    \includegraphics[height=1cm]{nih.png}\hspace{0.5cm}
    \includegraphics[height=1cm]{hms.png}\hspace{0.5cm}
    \includegraphics[height=1cm]{bwh.png}\\
    \scriptsize{NIH T32-HLO7901, Training Program in Sleep, Circadian, \& Respiratory Neurobiology}}
\end{frame}


\metroset{background=dark}
\begin{frame}[noframenumbering]
    \emph{\bf{\huge{Open questions?}}}
\end{frame}
\metroset{background=light}


\end{document}










